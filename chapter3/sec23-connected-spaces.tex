\setcounter{section}{22} % 23
\section{Connected Spaces}
\label{sec-connected}
 
%%%%%%%%%%%%%%%%%%%%%%%%% 6 %%%%%%%%%%%%%%%%%%%%%%%%%

\setcounter{subsection}{5} % 6
\subsection{}

\subsubsection{Problem}
Let $A \subseteq X$. Show that if $C$ is a connected subspace of $X$ that intersects both $A$ and $X \setminus A$, then $C$ intersects $\Bd A$.

\subsubsection{Solution}
Let $P = \bar A \cap C$ and $Q = (\overline{X \setminus A}) \cap C$. $P$ and $Q$ are closed in $C$, and their union is $C$. $P$ and $Q$ are nonempty since $P$ intersects both $A$ and $X \setminus A$. Since $C$ is connected, they are not disjoint. So $P \cap Q = \bar A \cap \overline{X \setminus A} \cap C = \Bd A \cap C$ is nonempty.

%%%%%%%%%%%%%%%%%%%%%%%%% 8 %%%%%%%%%%%%%%%%%%%%%%%%%

\setcounter{subsection}{7} % 8
\subsection{}

\subsubsection{Problem}
Determine whether or not $\R^\omega$ is connected in the uniform topology.

\subsubsection{Solution}
$\R^\omega$ in the uniform topology is not connected. Let $A, B \subseteq \R^\omega$ be the set of bounded and unbounded sequences respectively. $A$ and $B$ are nonempty, disjoint, and their union is $\R^\omega$. Let $d$ be the uniform metric. Then for all $x \in A$, $B(x, \frac12) \subseteq A$. And similarly for all $x \in B$, $B(x, \frac12) \subseteq B$. Thus $A$ and $B$ are open.

%%%%%%%%%%%%%%%%%%%%%%%%% 9 %%%%%%%%%%%%%%%%%%%%%%%%%

\setcounter{subsection}{8} % 9
\subsection{}

\subsubsection{Problem}
Let $A$ be a proper subset of $X$, and let $B$ be a proper subset of $Y$. If $X$ and $Y$ are connected, show that
\[(X \times Y) \setminus (A \times B)\]
is connected.

\subsubsection{Solution}
Let $W = (X \times Y) \setminus (A \times B)$. Fix $x_0 \in X \setminus A$, and $y_0 \in Y \setminus B$. As subspaces of $W$, $\{x_0\} \times Y$ and $X \times \{y_0\}$ are homeomorphic to $Y$ and $X$ respectively, and so both are connected.  They have a point $x_0 \times y_0$ in common. So their union
\[Z = (\{x_0\} \times Y) \cup (X \times \{y_0\})\]
is connected as well. For any $x \times y \in W$, define the subspace of $W$, $T_{x \times y}$ as follows:
\[T_{x \times y} =
\begin{cases}
\{x\} \times Y &\textrm{, if } x \notin A \\
X \times \{y\} &\textrm{, otherwise}
\end{cases}
\]
$T_{x \times y}$ is indeed a subspace of $W$, because $x \notin A$ or $y \notin B$. $T_{x \times y}$ is homeomorphic to $X$ or $Y$, and so is connected.  Every $T_{x \times y}$ intersects $Z$, in $x \times y_0$ or $x_0 \times y$. Using Problem 3, the union of $Z$ with all the $T_{x\times y}$ is connected. This union is $W$, since every $x \times y$ in $W$ is in $T_{x \times y}$. Thus $W$ is connected.

%%%%%%%%%%%%%%%%%%%%%%%%% 11 %%%%%%%%%%%%%%%%%%%%%%%%%

\setcounter{subsection}{10} % 11
\subsection{}

\subsubsection{Problem}
Let $p : X \to Y$ be a quotient map. Show that if each set $p^{-1}(\{y\})$ is connected, and if $Y$ is connected, then $X$ is connected.

\subsubsection{Solution}
Suppose $X = A \cup B$, with $A$ and $B$ disjoint and open. We will show that at least one of $A$ and $B$ is empty.

For every $y \in Y$, $A \cap p^{-1}(\{y\})$ and $B \cap p^{-1}(\{y\})$ do not form a separation of $p^{-1}(\{y\})$, hence at least one of them is empty. So each $p^{-1}(\{y\})$ is a subset of $A$ or $B$. $A$ and $B$ are saturated with respect to $p$.

$A$ and $B$ are open and saturated, so $p(A)$ and $p(B)$ are open in $Y$. Since $A$ and $B$ are disjoint and saturated, $p(A)$ and $p(B)$ are disjoint. $p(A) \cup p(B) = Y$. Since $Y$ is connected, $p(A)$ and $p(B)$ do not form a separation of $Y$. Thus at least one of $p(A)$ and $p(B)$ is empty, and so at least one of $A$ and $B$ is empty. This shows that $X$ does not have a separation.

%%%%%%%%%%%%%%%%%%%%%%%%% 12 %%%%%%%%%%%%%%%%%%%%%%%%%

\setcounter{subsection}{11} % 12
\subsection{}

\subsubsection{Problem}
Let $Y \subseteq X$; let $X$ and $Y$ be connected. Show that if $A$ and $B$ form a separation of $X \setminus Y$, then $Y \cup A$ and $Y \cup B$ are connected.

\subsubsection{Solution}
First we prove the following lemma:
\begin{lemma}
Suppose $X$ is a space and $X = P \cup Q$. If $U \subseteq P \cap Q$ is open in both $P$ and $Q$, then it is open in $X$. The same holds with ``open'' replaced by ``closed''.
\end{lemma}
\begin{proof}
Suppose $U$ is open in both $P$ and $Q$. Then there exists $P'$ and $Q'$ open in $X$ such that $U = P \cap P' = Q \cap Q'$. Then
\begin{align*}
P' \cap Q' &= (P' \cap Q') \cap X \\
&= (P' \cap Q') \cap (P \cup Q) \\
&= (P' \cap Q' \cap P) \cup (P' \cap Q' \cap Q) \\
&= (U \cap Q') \cup (U \cap P') \\
&= U \cup U \\
&= U
\end{align*}
Thus $U = P' \cap Q'$ is open in $X$. The same proof works with ``open'' replaced by ``closed'' throughout.
\end{proof}

We now return to the problem. We will show that $Y \cup A$ is connected. The proof for $Y \cup B$ is similar. Suppose $C$ and $D$ form a separation of $Y \cup A$. Then since $Y$ is connected, $Y \subseteq C$ or $Y \subseteq D$. Without loss of generality, let's say $Y \subseteq C$. Since $Y \cup A = C \cup D$ and $C$ and $D$ are disjoint, we get $D \subseteq A$. $D$ is clopen in $Y \cup A$, so $D$ is clopen in $A$. $A$ is clopen in $X \setminus Y$, so $D$ is clopen in $X \setminus Y$ as well.

$D$ is clopen in both $X \setminus Y$ and $Y \cup A$. Their union is $X$, so by the above lemma $D$ is clopen in $X$. $D$ is nonempty, and $X \setminus D$ is nonempty as well (because it includes $B$). Thus $D$ and $X \setminus D$ form a separation of $X$, which is a contradiction.

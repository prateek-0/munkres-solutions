\setcounter{section}{19} % 20
\section{The Metric Topology}
\label{sec-topspace-metric}
 
%%%%%%%%%%%%%%%%%%%%%%%%% 4 %%%%%%%%%%%%%%%%%%%%%%%%%

\setcounter{subsection}{3} % 4
\subsection{}

\subsubsection{Problem}

Consider the product, uniform, and box topologies on $\R^\omega$.
\begin{enumerate}
\item In which topologies are the following functions from $\R$ to $\R^\omega$ continuous?
\begin{align*}
f(t) &= (t, 2t, 3t, \ldots) \\
g(t) &= (t, t, t, \ldots) \\
h(t) &= (t, \tfrac12 t, \tfrac13 t, \ldots)
\end{align*}

\item In which topologies do the following sequences converge?
\begin{align*}
w_1 &= (1,1,1,1,\ldots),&  x_1 &=(1,1,1,1,\ldots), \\
w_2 &= (0,2,2,2,\ldots),&  x_2 &=(0,\tfrac12,\tfrac12,\tfrac12,\ldots), \\
w_3 &= (0,0,3,3, \ldots),& x_3 &=(0,0,\tfrac13,\tfrac13,\ldots), \\
&\cdots & &\cdots \\
y_1 &= (1,0,0,0,\ldots),& z_1 &=(1,1,0,0,\ldots), \\
y_2 &= (\tfrac12,\tfrac12,0,0,\ldots),& z_2 &=(\tfrac12,\tfrac12,0,0,\ldots), \\
y_3 &= (\tfrac13,\tfrac13,\tfrac13,0,\ldots),& z_3 &=(\tfrac13,\tfrac13,0,0,\ldots), \\
&\cdots & &\cdots
\end{align*}
\end{enumerate}

\subsubsection{Solution}

\begin{enumerate}
\item In the product topology, $f$, $g$, and $h$ are all continuous, as they are componentwise continuous.

In the box topology, none of them is continuous: Consider the open set $U \subseteq \R^\omega$ defined by
\[U = \prod_{n=1}^\infty \left(-\tfrac{1}{n^2}, \tfrac{1}{n^2} \right)\]
Then $f^{-1}(U) = g^{-1}(U) = h^{-1}(U) = \{0\}$, which is not open in $\R$.

Now let us consider the uniform topology. Let $d$ be the uniform metric on $\R^\omega$. We show that $g$ and $h$ are continuous, and the same argument works for both. Let $q \in \{g,h\}$. Let $t \in \R$ and let $U$ be a neighbourhood of $q(t)$. Let $\epsilon > 0$ such that $B(q(t), \epsilon) \subseteq U$. Then $q$ maps the open set $(t-\epsilon, t+\epsilon)$ into $B(q(t), \epsilon)$. So $q$ is continuous.

$f$ is not continuous in the uniform topology: Consider the ball in $\R^\omega$ of radius $\frac12$ around the origin. Its inverse image under $f$ is $\{0\}$, which is not open.

\item In this problem, if $x \in \R^\omega$, we use $x[1]$, $x[2]$, etc to refer to its components, to make it easier to refer to, for eample, $w_3[5]$. Let $\bfzero$ be the point in $\R^\omega$ with all components $0$.

In the \emph{product} topology, all four sequences converge to $\bfzero$. It's easy to see that for all basis neighbourhoods $U$ of $\bfzero$, all four sequences have only finitely many elements outside $U$. Further, since $\R^\omega$ is Hausdorff in the product topology, the sequences don't converge to any point other than $\bfzero$. Thus even in the uniform or box topologies (being finer than the product topology), the sequences cannot converge to any point other than $\bfzero$.

In the \emph{box} topology, $z$ converges to $\bfzero$, since given any box neighbourhood $U = \prod_{n=1}^\infty U_n$, consider $U_1 \cap U_2$, which is a neighbourhood of $0$ in $\R$. For all large enough $n$, $1/n \in U_1 \cap U_2$, and hence $z_n \in U$.

In the box topology, none of the other sequences converge. Consider the box neighbourhood of $\bfzero$,
\[U = \prod_{n=1}^\infty \left(-\tfrac{1}{n^2}, \tfrac{1}{n^2} \right)\]
Then for all $n$, no $x_n$, $y_n$, or $z_n$ is in $U$: $w_n[n] = n$ and $x_n[n] = y_n[n] = \frac1n$, neither of which is in $(-\frac{1}{n^2}, \frac{1}{n^2})$.

In the \emph{uniform} topology,
\[d(x_n,\bfzero) = d(y_n,\bfzero) = d(z_n, \bfzero) = \tfrac1n\]
and so $x$, $y$, and $z$ converge to $\bfzero$. $d(w_n, \bfzero) = 1$, and so $w$ is not eventually in the ball $B(\bfzero, \frac12)$. $w$ does not converge to $\bfzero$.
\end{enumerate}

%%%%%%%%%%%%%%%%%%%%%%%%% 5 %%%%%%%%%%%%%%%%%%%%%%%%%

\setcounter{subsection}{4} % 5
\subsection{}

\subsubsection{Problem}
Let $\R^\infty$ be the subset of $\R^\omega$ consisting of all sequences which are eventually $0$. What is the closure of $\R^\infty$ in $\R^\omega$ in the uniform topology? Justify your answer.

\subsubsection{Solution}
Let $C \subseteq \R^\omega$ be the set of all sequences which converge to $0$. We show that the closure of $\R^\infty$ in $\R^\omega$ in the uniform topology is $C$. Let $d$ be the uniform metric.

First we show that $C \subseteq \overline{R^\infty}$. Let $x \in C$. It suffices to show that every ball centred at $x$ intersects $\R^\infty$. Consider a ball $B(x, \epsilon)$. Let $N$ be such that for all $n \geq N$, $|x_n| < \epsilon/2$. Let $y \in \R^\infty$, such that $y_n = x_n$ for $n < N$ and $y_n = 0$ otherwise. Then $d(x,y) < \epsilon$, so $B(x, \epsilon)$ intersects $\R^\infty$. So $x \in \overline{\R^\infty}$.

Now we show that $\overline{R^\infty} \subseteq C$. Let $x \in \overline{R^\infty}$. We need to show that $x$ converges to $0$. Let $\epsilon > 0$. Then $B(x, \epsilon)$ intersects $\overline{\R^\infty}$. Let $y$ be an element in the intersection. $y$ is eventually zero, and the distance between $x$ and $y$ in the uniform metric is at most $\epsilon$, therefore $x$ is eventually in $(-\epsilon, \epsilon)$. As this holds for every $\epsilon$, $x$ converges to $0$.


%%%%%%%%%%%%%%%%%%%%%%%%% 8 %%%%%%%%%%%%%%%%%%%%%%%%%

\setcounter{subsection}{7} % 8
\subsection{}

\subsubsection{Problem}
Let $X$ be the subset of $\R^\omega$ consisting of all sequences $x$ such that $\sum x_i^2$ converges. Then the formula
\[d(x,y) = \sqrt{\sum_{i=1}^\infty (x_i - y_i)^2}\]
defines a metric on $X$ (see Exercise 10). On $X$ we have the three topologies it inherits from the box, uniform, and product topologies on $\R^\omega$. We also have the topology given by the metric $d$, which we call the $\ell^2$-topology.
\begin{enumerate}
\item Show that on $X$, we have the inclusions
\[\textrm{box topology} \supseteq \ell^2 \textrm { topology} \supseteq \textrm{ uniform topology}\]
\item The set $\R^\infty$ of all sequences that are eventually zero is a subset of $X$. Show that the four topologies that $\R^\infty$ inherits as a subspace of $X$ are all distinct.
\item The set
\[H = \prod_{n \in \Z^+} [0, \tfrac1n]\]
is a subset of $X$; it is called the \emph{Hilbert cube}. Compare the four topologies $H$ inherits as a subspace of $X$. Compare the four topologies $H$ inherits as a subspace of $X$.
\end{enumerate}

\subsubsection{Solution}
We use $d_1$ for the uniform metric and $d_2$ for the $\ell^2$ metric on $X$. Likewise we use $B_1$ and $B_2$ for the corresponding balls in $X$.
\begin{enumerate}
\item First we show that the $\ell^2$ topology is finer than the uniform topology. It is easy to see that for all $x, y \in X$, $d_1(x,y) \leq d_2(x,y)$.
Let $U \subseteq X$ be open in the uniform topology. We show that $U$ is open in the $\ell^2$ topology. Let $x \in U$. Then there is an $\epsilon > 0$ such that $B_1(x, \epsilon) \subseteq U$. It follows that $x \in B_2(x, \epsilon) \subseteq B_1(x, \epsilon) \subseteq U$. $U$ is open in the $\ell^2$ topology.

Now we show that the box topology is finer than the $\ell^2$ topology. Let $U \subseteq X$ be open in the $\ell^2$ topology. Let $x \in U$. There exists $\epsilon > 0$ such that $B_2(x, \epsilon) \subseteq U$. Consider the box neighbourhood of $x$, given by
\[V = X \cap \prod_{n \in \Z^+} \left(x_n-\tfrac{\epsilon}{2^n},x_n+\tfrac{\epsilon}{2^n}  \right)\]
We show that $V \subseteq B_2(x, \epsilon)$. If $y \in V$, then

\[
d_2(x,y) = \sqrt{\sum_{i=1}^\infty (x_i - y_i)^2} 
\leq \sqrt{\sum_{i=1}^\infty \frac{\epsilon^2}{2^{2n}}} 
= \epsilon\sqrt{\sum_{i=1}^\infty \frac{1}{4^n}} 
= \frac{\epsilon}{\sqrt 3}
< \epsilon
\]
Thus $x \in V \subseteq B_2(x, \epsilon) \subseteq U$. $U$ is open in the box topology.

\item We will that the \emph{box topology} is stricly finer than the $\ell^2$ \emph{topology} on $\R^\infty$. Consider the open set in the box topology
\[U_1 = \R^\infty \cap \prod_{n \in \Z^+} \left(-\tfrac{1}{2^n}, \tfrac{1}{2^n}\right)\]
Then $\bfzero \in U_1$. For $\epsilon > 0$, consider $B_2(\bfzero, \epsilon)$. Let $m \in \Z^+$ be such that $1/2^m < \epsilon/2$. Consider $x \in \R^\infty$ such that $x_m = \epsilon/2$ and $x_n = 0$ for all $n \neq m$. Then $d_2(\bfzero, x) = \epsilon/2$, so $x \in B(\bfzero, \epsilon)$. But $x \notin U_1$. No $\epsilon$-ball around $\bfzero$ is a subset of $U_1$, so $U_1$ is not open in the $\ell^2$ topology.

We will show that the $\ell^2$ \emph{topology} is strictly finer than the \emph{uniform topology} on $\R^\infty$. Consider $B_2(\bfzero, 1)$. We will show that this is not open in the uniform topology. For any $\epsilon > 0$, consider $B_1(\bfzero, \epsilon)$. Pick $N \in \Z^+$ such that $\frac{\sqrt{N}\epsilon}{2} > 1$. Let $x \in \R^\infty$ such that $x_n = \frac{\epsilon}{2}$ for $n \leq N$ and $x_n = 0$ otherwise. Then $d_1(x, \bfzero) \leq \epsilon/2$, so $x \in B_1(\bfzero, \epsilon)$. But $d_2(x, \bfzero) = \frac{\sqrt{N}\epsilon}{2} > 1$, so $x \notin B_2(\bfzero, 1)$. No uniform ball centred at $\bfzero$ is a subset of $B_2(0,1)$, so the latter is not open in the uniform topology.

We will show that the \emph{uniform topology} is strictly finer than the \emph{product topology} on $\R^\infty$. Consider $B_1(\bfzero, \frac12)$. We will show that this is not open in the product topology. Consider any basis neighbourhood $ \R^\infty \cap \prod_{n \in \Z^+} U_n$ of $\bfzero$ in the product topology, with $U_n = \R$ for all but finitely many $n$. Pick $m$ such that $U_m = \R$, and let $x \in \R^\infty$ such that $x_m = 1$ and $x_n = 0$ for all $n \neq m$. Then $x \in \R^\infty \cap \prod_{n \in \Z^+} U_n$, but $x \notin B_1(\bfzero, \frac12)$. Every basis neighbourood (in the product topology) of $\bfzero$ has an element outside $B_1(\bfzero, \frac12)$, so the latter is not open in the product topology.

\item First note that $H$ is indeed a subset of $X$, since $\sum_{n \in \Z^+} \frac{1}{n^2}$ converges.

We will show that the \emph{box topology} is strictly finer than the $\ell^2$ \emph{topology} on $H$. This argument is almost the same as on $\R^\infty$.


Consider the open set in the box topology
\[U = \prod_{n \in \Z^+} \left[0, \tfrac{1}{n+1}\right)\]
Then $\bfzero \in U$. For $\epsilon > 0$, consider $B_2(\bfzero, \epsilon)$. Let $m \in \Z^+$ be such that $1/(m+1) < \epsilon/2 \leq 1/m$. Consider $x \in H$ such that $x_m = \epsilon/2$ and $x_n = 0$ for all $n \neq m$. Then $d_2(\bfzero, x) = \epsilon/2$, so $x \in B_2(\bfzero, \epsilon)$. But $x \notin U$. No $\epsilon$-ball around $\bfzero$ is a subset of $U$, so $U$ is not open in the $\ell^2$ topology.

We know that
\[\ell^2 \textrm { topology} \supseteq \textrm{uniform topology} \supseteq \textrm{product topology}\]
on $\R^\omega$ and so the same inclusions hold when restricted to the subspace $H$. Further we show that on $H$,
\[\textrm{product topology} \supseteq \ell^2 \textrm { topology}\]
Let $U \subseteq H$ be open in the $\ell^2$ topology, and let $x \in U$. Then there is an $\epsilon > 0$ such that $B_2(x, \epsilon) \subseteq U$. Pick $N$ such that
\[\sum_{n=N+1}^\infty \frac{1}{n^2} < \frac{\epsilon^2}{2}\]

Consider the neighbourhood of $x$ (in $H$ in the product topology) given by $ V = H \cap \prod_{n \in \Z^+} U_n$, where
\[U_n =
\begin{cases}
\left(x_n - \tfrac{\epsilon}{\sqrt{2N}}, x_n + \tfrac{\epsilon}{\sqrt{2N}} \right) & \textrm{, if } n \leq N \\
\R & \textrm{, if } n > N
\end{cases}
\]
Then for any $y \in V$,
\begin{align*}
d_2(x,y) &= \sqrt{\sum_{n\in\Z^+} (x_n-y_n)^2} \\
&= \sqrt{\sum_{n=1}^N (x_n-y_n)^2 + \sum_{n=N+1}^\infty (x_n-y_n)^2} \\
&\leq \sqrt{N \cdot \frac{\epsilon^2}{2N} + \sum_{n=N+1}^\infty \frac{1}{n^2}} \\
&\leq \sqrt{\frac{\epsilon^2}{2} + \frac{\epsilon^2}{2}} \\
&=\epsilon
\end{align*}
Thus $x \in V \subseteq B_2(x,\epsilon) \subseteq U$. $U$ is open in the product topology.

Thus the product topology is finer than the $\ell^2$ topology on $H$, and so the product, uniform, and $\ell^2$ topologies are equal on $H$.
\end{enumerate}

%%%%%%%%%%%%%%%%%%%%%%%%% 9 %%%%%%%%%%%%%%%%%%%%%%%%%

\setcounter{subsection}{8} % 9
\subsection{}

\subsubsection{Problem}
Show that the euclidean metric $d$ on $\R^n$ is a metric, as follows: If $\textbf{x}, \textbf{y} \in \R^n$ and $c \in \R$, defined
\begin{eqnarray*}
    \textbf{x} + \textbf{y} & = & (x_1+y_1, \dots, x_n+y_n),\\
    c\textbf{x} & = & (cx_1, \dots, cx_n),\\
    \textbf{x}\cdot\textbf{y} & = & x_1y_1 + \dots + x_ny_n.
\end{eqnarray*}
\begin{enumerate}
    \item Show that $\textbf{x}\cdot(\textbf{y} + \textbf{z}) = (\textbf{x}\cdot\textbf{y}) + (\textbf{x}\cdot\textbf{z})$.
    \item Show that $|\textbf{x}\cdot\textbf{y}| \leq \|\textbf{x}\|\|\textbf{y}\|$. [\emph{Hint:} If $\textbf{x}, \textbf{y} \neq 0$, let $a := 1/\|x\|$ and $b := 1/\|y\|$, and use the fact that $\|a\textbf{x} \pm b\textbf{y}\| \geq 0$.]
    \item Show that $\|\textbf{x} + \textbf{y}\| \leq \|\textbf{x}\| + \|\textbf{y}\|$. [\emph{Hint:} Compute $(\textbf{x} + \textbf{y})\cdot(\textbf{x} + \textbf{y})$ and apply $(b)$.]
    \item Verify that $d$ is a metric.
\end{enumerate}

\subsubsection{Solution}
\todo




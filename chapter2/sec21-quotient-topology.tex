\setcounter{section}{20} % 21
\section{The Quotient Topology}
\label{sec-topspace-quotient}
 
%%%%%%%%%%%%%%%%%%%%%%%%% 2 %%%%%%%%%%%%%%%%%%%%%%%%%

\setcounter{subsection}{1} % 2
\subsection{}

\subsubsection{Problem}
\begin{enumerate}
\item Let $p : X \to Y$ be a continuous map. Show that if there is a continuous map $f : Y \to X$ such that $p \circ f$ equals the identity map of $Y$, then $p$ is a quotient map.
\item If $ A \subseteq X$, a retraction of $X$ onto $A$ is a continuous map $r : X \to A$ such that $r(a) = a$ for each $a \in A$. Show that every retraction is a quotient map.
\end{enumerate}

\subsubsection{Solution}
\begin{enumerate}
\item $p$ is given to be continuous, and for every $y \in Y$, $p(f(y)) = y$, so $p$ is surjective. Let $i_Y$ be the identity map on $Y$. Consider $U \subseteq Y$, such that $p^{-1}(U)$ is open. $f^{-1}(p^{-1}(U))$ is open, because $f$ is continuous. But
\[f^{-1}(p^{-1}(U)) = (p \circ f)^{-1}(U) = i_Y^{-1}(U) = U\]
So $U$ is open whenever $p^{-1}(U)$ is open. $p$ is a quotient map.
\item Use the inclusion map from $A$ to $X$ along with the above.
\end{enumerate}

%%%%%%%%%%%%%%%%%%%%%%%%% 3 %%%%%%%%%%%%%%%%%%%%%%%%%

\setcounter{subsection}{2} % 3
\subsection{}

\subsubsection{Problem}
Let $\pi_1 : \R \times \R \to \R$ be the projection on the first coordinate. Let $A$ be the subspace of $\R \times \R$ consisting of all points $x \times y$ for which either $x \geq 0$ or $y = 0$ (or both). Let $q : A \to \R$ be obtained by restricting $\pi_1$. Show that $q$ is a quotient map that is neither open nor closed.

\subsubsection{Solution}
Consider $f : \R \to A$ given by $f(x) = x \times 0$. Then $q \circ f$ is the identity on $\R$, so $f$ is a quotient map.

Consider $U = (-1, 1) \times (1,2)$, which is open in $\R \times \R$. $U \cap A = [0,1) \times (1,2)$ is open in $A$. $q(U) = [0,1)$ is not open in $\R$, so $q$ is not an open map.

Consider $C = \{ x \times y : x \times y \in A, xy=1\}$. $C$ is closed in $A$, but $q(C) = (0, \infty)$ is not closed in $\R$. So $q$ is not a closed map.

%%%%%%%%%%%%%%%%%%%%%%%%% 4 %%%%%%%%%%%%%%%%%%%%%%%%%

\setcounter{subsection}{3} % 4
\subsection{}

\subsubsection{Problem}
\begin{enumerate}
\item Define an equivalence relation on the plane $X = \R^2$ as follows:
\[x_0 \times y_0 \sim x_1 \times y_1 \textrm{ if } x_0 + y_0^2 = x_1 + y_1^2\]
Let $X^*$ be the corresponding quotient space. It is homeomorphic to a familiar space; what is it? [Hint: Set $g(x \times y) = x + y^2$]
\item Repeat the above for the equivalence relation
\[x_0 \times y_0 \sim x_1 \times y_1 \textrm{ if } x_0^2 + y_0^2 = x_1^2 + y_1^2\]
\end{enumerate}

\subsubsection{Problem}
\begin{enumerate}
\item Let $g : \R^2 \to \R$, given by $g(x \times y) = x + y^2$. $g$ is continuous. Let $f : \R \to \R^2$ given by $f(x) = x \times 0$. $f$ is also continuous, and $g \circ f$ is the identity function on $\R$. So $g$ is a quotient map.

By Corollary 22.3, $g$ induces a homeomorphism from $X^*$ to $\R$.

\item Let $Z$ be $[0,\infty)$ as a subspace of $\R$. Let $g : \R^2 \to Z$, given by $g(x \times y) = x^2 + y^2$. $g$ is continuous. Let $f : \R \to \R^2$ given by $f(x) = \sqrt{x} \times 0$. $f$ is also continuous, and $g \circ f$ is the identity function on $Z$. So $g$ is a quotient map.

By Corollary 22.3, $g$ induces a homeomorphism from $X^*$ to $Z$.
\end{enumerate}

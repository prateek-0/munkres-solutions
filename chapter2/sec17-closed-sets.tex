\setcounter{section}{16} % 17
\section{Closed Sets and Limit Points}
\label{sec-topspace-closed}

%%%%%%%%%%%%%%%%%%%%%%%%% 2 %%%%%%%%%%%%%%%%%%%%%%%%%

\setcounter{subsection}{1} % 2
\subsection{}

\subsubsection{Problem}
Show that if $A$ is closed in $Y$ and $Y$ is closed in $X$, then $A$ is closed in $X$.
\subsubsection{Solution}
From Theorem 17.2 we get that there exists a closed subset $C \subseteq X$ such that $A = C \cap Y$. Since $Y$ is closed in $X$, from Theorem 17.1(2), we get that $A = (C \cap Y) \subseteq X$ is also a closed subset of $X$.

%%%%%%%%%%%%%%%%%%%%%%%%% 3 %%%%%%%%%%%%%%%%%%%%%%%%%

\setcounter{subsection}{2} % 3
\subsection{}

\subsubsection{Problem}
Show that if $A$ is closed in $X$ and $B$ is closed in $Y$, then $A \times B$ is closed in $X \times Y$.
\subsubsection{Solution}
We observe the following identity:
\[ A \times B = (X \times Y) \setminus ((X\setminus A) \times Y \cup X \times (Y \setminus B)) \]
Hence the problem reduces to showing that $((X\setminus A) \times Y \cup X \times (Y \setminus B))$ is open in $X \times Y$.
By definition of closed sets, we get that $A$ closed in $X \implies X\setminus A$ is open in $X$ and $B$ closed in $Y \implies Y\setminus B$ is open in $Y$. Hence, $(X\setminus A) \times Y$ and $X \times (Y \setminus B)$ are open in $X \times Y$ and hence $((X\setminus A) \times Y \cup X \times (Y \setminus B))$ is an open subset of $X \times Y$.

%%%%%%%%%%%%%%%%%%%%%%%%% 4 %%%%%%%%%%%%%%%%%%%%%%%%%

\setcounter{subsection}{3} % 4
\subsection{}

\subsubsection{Problem}
Show that if $U$ is open in $X$ and $A$ is closed in $X$, then $U-A$ is open in $X$, and $A-U$ is closed in $X$.
\subsubsection{Solution}
By definition, we have $U$ open in $X \implies X\setminus U$ is closed in $X$ and $A$ closed in $X \implies X\setminus A$ is open in $X$. Since finite intersections of open sets are open, we have that $U-A := U \cap (X\setminus A)$ is open in $X$. Similarly, since arbitrary intersections of closed sets are closed, we have $A-U := A \cap (X\setminus U)$ is closed in $X$.

%%%%%%%%%%%%%%%%%%%%%%%%% 6 %%%%%%%%%%%%%%%%%%%%%%%%%

\setcounter{subsection}{5} % 6
\subsection{}

\subsubsection{Problem}
Let $A$, $B$, and $A_\alpha$ denote subsets of a space $X$. Prove the following:
\begin{enumerate}
 \item If $A \subseteq B$, then $\bar{A} \subseteq \bar{B}$.
 \item $\overline{A \cup B} = \bar A \cup \bar B$.
 \item $\overline{\bigcup A_\alpha} \supseteq \bigcup \overline{A_\alpha}$. Give an example where equality fails.
\end{enumerate}
\subsubsection{Solution}
\begin{enumerate}
 \item We have $A \subseteq B \subseteq \bar B$, and $\bar B$ is a closed set. Hence $\bar A \subseteq \bar B$.
 \item Clearly $A \cup B \subseteq \bar A \cup \bar B$, and the latter is closed, hence $\overline{A \cup B} \subseteq \bar A \cup \bar B$. Conversely, clearly $\bar A \subseteq \overline{A \cup B}$ and $\bar B \subseteq \overline{A \cup B}$. So $\bar A \cup \bar B \subseteq \overline{A \cup B}$.
 \item Let $Y = \bigcup A_\alpha$. Then for each $\alpha$, $A_\alpha \subseteq Y$, and so $\overline{A_\alpha} \subseteq \bar Y$. So $\bigcup \overline{A_\alpha} \subseteq \bar Y$.
 
 For an example where equality fails, let $X = \R$ with the standard topology. Treating $\Q$ as an index set, let $A_\alpha = \{\alpha\}$ for $\alpha \in \Q$. Then $\overline{\bigcup A_\alpha} = \bar \Q = \R$. And $\bigcup \overline{A_\alpha} = \Q$.
 \end{enumerate}

%%%%%%%%%%%%%%%%%%%%%%%%% 8 %%%%%%%%%%%%%%%%%%%%%%%%%
 
\setcounter{subsection}{7} % 8
\subsection{}
\subsubsection{Problem}
Let $A$, $B$, and $A_\alpha$ denote subsets of a space $X$. Determine whether the following equations hold; if an equality fails, determine whether one of the inclusions $\subseteq$ or $\supseteq$ holds.
\begin{enumerate}
 \item $\overline{A \cap B} = \bar A \cap \bar B$.
 \item $\overline{\bigcap A_\alpha} = \bigcap \overline{A_\alpha}$
 \item $\overline{A \setminus B} = \bar A \setminus \bar B$
\end{enumerate}
 
\subsubsection{Solution}
\begin{enumerate}
 \item We have $\overline{A \cap B} \subseteq \bar A$ and $\overline{A \cap B} \subseteq \bar B$, hence $\overline{A \cap B} \subseteq \bar A \cap \bar B$. The reverse inclusion does not hold, for example let $X = \R$ with the standard topology, $A = (0,1)$, and $B = (1,2)$.
 
 \item The same as the above, taking an index set of size 2.
 
 \item We show that $\bar A \setminus \bar B \subseteq \overline{A \setminus B}$. Let $x \in \bar A \setminus \bar B$. We show that every neighbourhood of $x$ intersects $A \setminus B$. Fix a neighbourhood $U_0$ of $x$ such that $U_0 \subseteq X \setminus \bar B$. Let $U$ be an arbitrary neighbourhood of $x$. Then since $x \in \bar A$, $U \cap U_0$, which is also a neighbourhood of $x$, intersects $A$. Let $y \in U \cap U_0 \cap A$. Then clearly $y \in U \cap (A \setminus B)$, which shows that every neighbourhood of $x$ intersects $A \setminus B$.
 
 The converse, $\overline{A \setminus B} \subseteq \bar A \setminus \bar B$, does not hold. For example, let $X = \R$ with the standard topology, $A = [0,1]$, and $B = (0,1)$. 
\end{enumerate}

%%%%%%%%%%%%%%%%%%%%%%%%% 9 %%%%%%%%%%%%%%%%%%%%%%%%%

\setcounter{subsection}{8} % 9
\subsection{}
\subsubsection{Problem}
Let $A \subseteq X$ and $B \subseteq Y$. Show that in the space $X \times Y$,
\[\overline{A \times B} = \bar A \times \bar B \]
\subsubsection{Solution}
$A \times B$ is a subset of the closed set $\bar A \times \bar B$, and so $\overline{A \times B} \subseteq \bar A \times \bar B$. Conversely, let $x \times y \in \bar A \times \bar B$. Consider any basis neighbourhood of it in $X \times Y$, $U \times V$. Then $U$ intersects $A$ and $V$ intersects $B$, so $U \times V$ intersects $A \times B$. So $x \times y \in \overline{A \times B}$. 

%%%%%%%%%%%%%%%%%%%%%%%%% 13 %%%%%%%%%%%%%%%%%%%%%%%%%

\setcounter{subsection}{12} % 13
\subsection{}
\subsubsection{Problem}
Show that $X$ is Hausdorff if and only if the diagonal $\Delta = \{x \times x \mid x \in X\}$ is closed in $X \times X$.
\subsubsection{Solution}
Let $\T$ be the toplogy on $X$ and let $\Omega = (X \times X) \setminus \Delta$.
\begin{align*}
& X \textrm{ is Hausdorff} \\
\iff& \forall x,y \in X \ (x \neq y \implies \exists U, V \in \T (x \in U, y \in V, U \cap V = \emptyset)) \\
\iff& \forall (x \times y) \in \Omega \ ( \exists U, V \in \T (x \in U, y \in V, U \cap V = \emptyset)) \\
\iff& \forall (x \times y) \in \Omega \ ( \exists U, V \in \T ( x \times y \in U \times V \subseteq \Omega)) \\
\iff& \Omega \textrm{ is open} \\
\iff& \Delta \textrm{ is closed}
\end{align*}

%%%%%%%%%%%%%%%%%%%%%%%%% 19 %%%%%%%%%%%%%%%%%%%%%%%%%

\setcounter{subsection}{18} % 19
\subsection{}
\subsubsection{Problem}
If $A \subseteq X$, we define the boundary of $A$ by the equation
\[\Bd A = \bar A \cap \overline{X \setminus A}\]
\begin{enumerate}
\item Show that $\Int A$ and $\Bd A$ are disjoint, and $\bar A = \Int A \cup \Bd A$.
\item Show that $\Bd A = \emptyset \iff A$ is both open and closed.
\item Show that $U$ is open $\iff \Bd U = \bar U \setminus U$.
\item If $U$ is open, is it true that $U = \Int(\bar U)$? Justify your answer.
\end{enumerate}

\subsubsection{Solution}
\begin{enumerate}
\item $X \setminus A$ is a subset of the closed set $X \setminus \Int A$, hence $\overline{X \setminus A} \subseteq X \setminus \Int A$. Thus $\Bd A \subseteq \overline{X \setminus A} \subseteq X \setminus \Int A$. $\Bd A$ and $\Int A$ are disjoint.

Let $x \in \bar A$. If every neighbourhood of $x$ intersects $X \setminus A$, then $x \in \overline{X \setminus A}$, and hence $x \in \Bd A$. Otherwise, there is some neighbourhood of $x$ which does not intersects $X \setminus A$, that is, it is a subset of $A$, and so $x \in \Int A$. Thus $\bar A = \Int A \cup \Bd A$.

\item If $\Bd A = \emptyset$, then $\bar A$ and $\overline{X \setminus A}$ form a partition of $X$. Thus $A = \bar A$ and $X \setminus A = \overline{X \setminus A}$. Both $A$ and $X \setminus A$ are closed.

Conversely, if $A$ is both open and closed, then $X \setminus A$ is closed as well, and so $\Bd A = A \cap (X \setminus A) = \emptyset$.

\item For all $A \subseteq X$, we know that $\Int A$ and $\Bd A$ form a partition of $\bar A$, and hence $\Bd A = \bar A \setminus \Int A$. Then, noting that $A$ and $\Int A$ are always subsets of $\bar A$,
\begin{align*}
& U \textrm{ is open} \\
\iff& \Int U = U \\
\iff& \bar U \setminus \Int U = \bar U \setminus U \\
\iff& \Bd U = \bar U \setminus U
\end{align*}

\item No. Let $X = \R$ with the standard topology, and let $U = (0,1) \cup (1,2)$.
\end{enumerate}


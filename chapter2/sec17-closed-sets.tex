\setcounter{section}{16} % 17
\section{Closed Sets and Limit Points}
\label{sec-topspace-closed}

%%%%%%%%%%%%%%%%%%%%%%%%% 1 %%%%%%%%%%%%%%%%%%%%%%%%%

\setcounter{subsection}{0} % 1
\subsection{}

\subsubsection{Problem}
Let $\mathcal{C}$ be a collection of subsets of the set $X$. Suppose that $\emptyset$ and $X$ are in $\mathcal{C}$, and that finite unions and arbitrary intersections of elements of $\mathcal{C}$ are in $\mathcal{C}$. Show that the collection
\[ \T := \left\{ X-C \mid C \in \mathcal{C} \right\} \]
is a topology on $X$.
\subsubsection{Solution}
Following the definition of a topology on a set, we see that
\begin{enumerate}
    \item $X \in \mathcal{C} \implies \emptyset = X-X \in \T$ and $\emptyset \in \mathcal{C} \implies X = X-\emptyset \in \T$.
    \item Let $\T_0 := \{T_i \mid i \in I\}$ be a subcollection of elements in $\T$ where $I$ is some indexing set. Then for each set $T_i \in \T_0$ there exists a corresponding set $C_i \in \mathcal{C}$ such that $T_i := X - C_i$ for all $i \in I$. Since $\mathcal{C}$ is given to be closed under arbitrary intersections, there exists $C \in \mathcal{C}$ such that $C := \bigcap_{i \in I}C_i$. Then by De Morgan's Laws we get that
    \[ \bigcup_{i \in I}T_i := \bigcup_{i \in I}(X-C_i) = X-\left( \bigcap_{i \in I}C_i \right)  = X-C \in \T \]
    \item Let $\T_1 := \{ T_1, T_2, \dots, T_n \}$ be a finite subcollection of elements of $\T$ for some $n \in \N$. Then there exist sets $C_1, C_2, \dots, C_n \in \mathcal{C}$ such that $T_i := X-C_i$ for all $i \in [n]$. Since $\mathcal{C}$ is given to be closed under finite unions, there exists $C' \in \mathcal{C}$ such that $C' := C_1 \cup C_2 \cup \dots \cup C_n$. Then by De Morgan's Laws we get that
    \[ \bigcap_{i=1}^n T_i := \bigcap_{i=1}^n (X-C_i) = X-\left( \bigcup_{i=1}^n C_i \right) = X-C' \in \T \]
\end{enumerate}
Since $\T$ satisfies all mentioned conditions, we conclude that $\T$ is a topology over $X$. 

%%%%%%%%%%%%%%%%%%%%%%%%% 2 %%%%%%%%%%%%%%%%%%%%%%%%%

\setcounter{subsection}{1} % 2
\subsection{}

\subsubsection{Problem}
Show that if $A$ is closed in $Y$ and $Y$ is closed in $X$, then $A$ is closed in $X$.
\subsubsection{Solution}
From Theorem 17.2 we get that there exists a closed subset $C \subseteq X$ such that $A = C \cap Y$. Since $Y$ is closed in $X$, from Theorem 17.1(2), we get that $A = (C \cap Y) \subseteq X$ is also a closed subset of $X$.

%%%%%%%%%%%%%%%%%%%%%%%%% 3 %%%%%%%%%%%%%%%%%%%%%%%%%

\setcounter{subsection}{2} % 3
\subsection{}

\subsubsection{Problem}
Show that if $A$ is closed in $X$ and $B$ is closed in $Y$, then $A \times B$ is closed in $X \times Y$.
\subsubsection{Solution}
We observe the following identity:
\[ A \times B = (X \times Y) \setminus ((X\setminus A) \times Y \cup X \times (Y \setminus B)) \]
Hence the problem reduces to showing that $((X\setminus A) \times Y \cup X \times (Y \setminus B))$ is open in $X \times Y$.
By definition of closed sets, we get that $A$ closed in $X \implies X\setminus A$ is open in $X$ and $B$ closed in $Y \implies Y\setminus B$ is open in $Y$. Hence, $(X\setminus A) \times Y$ and $X \times (Y \setminus B)$ are open in $X \times Y$ and hence $((X\setminus A) \times Y \cup X \times (Y \setminus B))$ is an open subset of $X \times Y$.

%%%%%%%%%%%%%%%%%%%%%%%%% 4 %%%%%%%%%%%%%%%%%%%%%%%%%

\setcounter{subsection}{3} % 4
\subsection{}

\subsubsection{Problem}
Show that if $U$ is open in $X$ and $A$ is closed in $X$, then $U-A$ is open in $X$, and $A-U$ is closed in $X$.
\subsubsection{Solution}
By definition, we have $U$ open in $X \implies X\setminus U$ is closed in $X$ and $A$ closed in $X \implies X\setminus A$ is open in $X$. Since finite intersections of open sets are open, we have that $U-A := U \cap (X\setminus A)$ is open in $X$. Similarly, since arbitrary intersections of closed sets are closed, we have $A-U := A \cap (X\setminus U)$ is closed in $X$.

%%%%%%%%%%%%%%%%%%%%%%%%% 5 %%%%%%%%%%%%%%%%%%%%%%%%%

\setcounter{subsection}{4} % 5
\subsection{}

\subsubsection{Problem}
Let $X$ be an ordered set in the order topology. Show that $\overline{(a, b)} \subseteq [a, b]$. Under what conditions does equality hold?
\subsubsection{Solution}
\todo

%%%%%%%%%%%%%%%%%%%%%%%%% 6 %%%%%%%%%%%%%%%%%%%%%%%%%

\setcounter{subsection}{5} % 6
\subsection{}

\subsubsection{Problem}
Let $A$, $B$, and $A_\alpha$ denote subsets of a space $X$. Prove the following:
\begin{enumerate}
 \item If $A \subseteq B$, then $\bar{A} \subseteq \bar{B}$.
 \item $\overline{A \cup B} = \bar A \cup \bar B$.
 \item $\overline{\bigcup A_\alpha} \supseteq \bigcup \overline{A_\alpha}$. Give an example where equality fails.
\end{enumerate}
\subsubsection{Solution}
\begin{enumerate}
 \item We have $A \subseteq B \subseteq \bar B$, and $\bar B$ is a closed set. Hence $\bar A \subseteq \bar B$.
 \item Clearly $A \cup B \subseteq \bar A \cup \bar B$, and the latter is closed, hence $\overline{A \cup B} \subseteq \bar A \cup \bar B$. Conversely, clearly $\bar A \subseteq \overline{A \cup B}$ and $\bar B \subseteq \overline{A \cup B}$. So $\bar A \cup \bar B \subseteq \overline{A \cup B}$.
 \item Let $Y = \bigcup A_\alpha$. Then for each $\alpha$, $A_\alpha \subseteq Y$, and so $\overline{A_\alpha} \subseteq \bar Y$. So $\bigcup \overline{A_\alpha} \subseteq \bar Y$.
 
 For an example where equality fails, let $X = \R$ with the standard topology. Treating $\Q$ as an index set, let $A_\alpha = \{\alpha\}$ for $\alpha \in \Q$. Then $\overline{\bigcup A_\alpha} = \bar \Q = \R$. And $\bigcup \overline{A_\alpha} = \Q$.
 \end{enumerate}

%%%%%%%%%%%%%%%%%%%%%%%%% 7 %%%%%%%%%%%%%%%%%%%%%%%%%

\setcounter{subsection}{6} % 7
\subsection{}

\subsubsection{Problem}
Criticize the following ``proof'' that $\overline{\bigcup A_\alpha} \subseteq \bigcup \overline{A_\alpha}$: if $\{A_\alpha\}$ is a collection of sets in $X$ and if $x \in \overline{\bigcup A_\alpha}$, then every neighbourhood $U$ of $x$ intersects $\bigcup A_\alpha$. Thus $U$ must intersect some $A_\alpha$, so that $x$ must belong to the closure of some $A_\alpha$. Therefore, $x \in \bigcup \overline{A_\alpha}$.
\subsubsection{Solution}
\todo

%%%%%%%%%%%%%%%%%%%%%%%%% 8 %%%%%%%%%%%%%%%%%%%%%%%%%
 
\setcounter{subsection}{7} % 8
\subsection{}
\subsubsection{Problem}
Let $A$, $B$, and $A_\alpha$ denote subsets of a space $X$. Determine whether the following equations hold; if an equality fails, determine whether one of the inclusions $\subseteq$ or $\supseteq$ holds.
\begin{enumerate}
 \item $\overline{A \cap B} = \bar A \cap \bar B$.
 \item $\overline{\bigcap A_\alpha} = \bigcap \overline{A_\alpha}$
 \item $\overline{A \setminus B} = \bar A \setminus \bar B$
\end{enumerate}
 
\subsubsection{Solution}
\begin{enumerate}
 \item We have $\overline{A \cap B} \subseteq \bar A$ and $\overline{A \cap B} \subseteq \bar B$, hence $\overline{A \cap B} \subseteq \bar A \cap \bar B$. The reverse inclusion does not hold, for example let $X = \R$ with the standard topology, $A = (0,1)$, and $B = (1,2)$.
 
 \item The same as the above, taking an index set of size 2.
 
 \item We show that $\bar A \setminus \bar B \subseteq \overline{A \setminus B}$. Let $x \in \bar A \setminus \bar B$. We show that every neighbourhood of $x$ intersects $A \setminus B$. Fix a neighbourhood $U_0$ of $x$ such that $U_0 \subseteq X \setminus \bar B$. Let $U$ be an arbitrary neighbourhood of $x$. Then since $x \in \bar A$, $U \cap U_0$, which is also a neighbourhood of $x$, intersects $A$. Let $y \in U \cap U_0 \cap A$. Then clearly $y \in U \cap (A \setminus B)$, which shows that every neighbourhood of $x$ intersects $A \setminus B$.
 
 The converse, $\overline{A \setminus B} \subseteq \bar A \setminus \bar B$, does not hold. For example, let $X = \R$ with the standard topology, $A = [0,1]$, and $B = (0,1)$. 
\end{enumerate}

%%%%%%%%%%%%%%%%%%%%%%%%% 9 %%%%%%%%%%%%%%%%%%%%%%%%%

\setcounter{subsection}{8} % 9
\subsection{}
\subsubsection{Problem}
Let $A \subseteq X$ and $B \subseteq Y$. Show that in the space $X \times Y$,
\[\overline{A \times B} = \bar A \times \bar B \]
\subsubsection{Solution}
$A \times B$ is a subset of the closed set $\bar A \times \bar B$, and so $\overline{A \times B} \subseteq \bar A \times \bar B$. Conversely, let $x \times y \in \bar A \times \bar B$. Consider any basis neighbourhood of it in $X \times Y$, $U \times V$. Then $U$ intersects $A$ and $V$ intersects $B$, so $U \times V$ intersects $A \times B$. So $x \times y \in \overline{A \times B}$. 

%%%%%%%%%%%%%%%%%%%%%%%%% 10 %%%%%%%%%%%%%%%%%%%%%%%%%

\setcounter{subsection}{9} % 10
\subsection{}

\subsubsection{Problem}
Show that every order topology is Hausdorff.
\subsubsection{Solution}
\todo

%%%%%%%%%%%%%%%%%%%%%%%%% 11 %%%%%%%%%%%%%%%%%%%%%%%%%

\setcounter{subsection}{10} % 11
\subsection{}

\subsubsection{Problem}
Show that the product of two Hausdorff spaces is Hausdorff.
\subsubsection{Solution}
\todo

%%%%%%%%%%%%%%%%%%%%%%%%% 12 %%%%%%%%%%%%%%%%%%%%%%%%%

\setcounter{subsection}{11} % 12
\subsection{}

\subsubsection{Problem}
Show that a subspace of a Hausdorff space is Hausdorff.
\subsubsection{Solution}
\todo

%%%%%%%%%%%%%%%%%%%%%%%%% 13 %%%%%%%%%%%%%%%%%%%%%%%%%

\setcounter{subsection}{12} % 13
\subsection{}
\subsubsection{Problem}
Show that $X$ is Hausdorff if and only if the diagonal $\Delta = \{x \times x \mid x \in X\}$ is closed in $X \times X$.
\subsubsection{Solution}
Let $\T$ be the toplogy on $X$ and let $\Omega = (X \times X) \setminus \Delta$.
\begin{align*}
& X \textrm{ is Hausdorff} \\
\iff& \forall x,y \in X \ (x \neq y \implies \exists U, V \in \T (x \in U, y \in V, U \cap V = \emptyset)) \\
\iff& \forall (x \times y) \in \Omega \ ( \exists U, V \in \T (x \in U, y \in V, U \cap V = \emptyset)) \\
\iff& \forall (x \times y) \in \Omega \ ( \exists U, V \in \T ( x \times y \in U \times V \subseteq \Omega)) \\
\iff& \Omega \textrm{ is open} \\
\iff& \Delta \textrm{ is closed}
\end{align*}

%%%%%%%%%%%%%%%%%%%%%%%%% 14 %%%%%%%%%%%%%%%%%%%%%%%%%

\setcounter{subsection}{13} % 14
\subsection{}

\subsubsection{Problem}
In the finite complement topology on $\R$, to what point or points does the sequence $x_n := 1/n$ converge?
\subsubsection{Solution}
\todo

%%%%%%%%%%%%%%%%%%%%%%%%% 15 %%%%%%%%%%%%%%%%%%%%%%%%%

\setcounter{subsection}{14} % 15
\subsection{}
\subsubsection{Problem}
Show the $T_1$ axiom is equivalent to the condition that for each pair of \emph{distinct} points of $X$, each has a neighbourhood not containing the other.

\subsubsection{Solution}
Consider the following  statements:
\begin{enumerate}
\item\label{t1} ($X$ satisfies $T_1$ axiom) All finite-point subsets of $X$ are closed in $X$.
\item\label{1t1} All singleton subsets of $X$ are closed in $X$.
\item\label{nbds} For each pair of distinct points in $X$, each has a neighbourhood not containing the other.
\end{enumerate}
For this problem, we need to show that $\ref{t1} \iff \ref{nbds}$. We will show that $\ref{t1} \iff \ref{1t1}$ and $\ref{1t1} \iff \ref{nbds}$.
Clearly, singleton sets are finite which gives us that $\ref{t1} \implies \ref{1t1}$. Conversely, given a finite subset $\set{x_1, x_2 \dots, x_n} \subseteq X$, each of $\set{x_i}, i \in \N_n := \set{1, 2, \dots, n}$ is closed, which implies that $\set{x_1, x_2, \dots, x_n} = \set{x_1} \cup \set{x_2} \cup \dots \cup \set{x_n}$ is also closed, being a finite union of closed sets. Hence $\ref{1t1} \implies \ref{t1}$.
\medskip

To show that $\ref{1t1} \implies \ref{nbds}$, let $x,y \in X, x \neq y$. $\ref{1t1}$ implies that $\set{x}, \set{y}$ are closed sets. Hence $X\setminus\set{x}, X\setminus\set{y}$ are open sets containing $y, x$ respectively. Also, $x \notin X\setminus\set{x}, y \notin X\setminus\set{y}$. Since $x, y$ were arbitrary points in $X$, we see that \ref{nbds} holds. Conversely, let $x_0 \in X$. \ref{nbds} implies that for all $y \in X, y \neq x_0$ there exists a neighbourhood $U_y$ of $y$ which does not contain $x_0$. Let $U := \bigcup_{y \in X\setminus\set{x_0}}U_y$. Then $U$ is open, being the union of open sets $U_y$ for $y \in X\setminus\set{x_0}$. Also, $U$ does not contain $x_0$, since $x_0 \notin U_y$ for all $y \in X\setminus\set{x_0}$. Finally for any $y \in X, y \neq x_0$ we have $y \in U_y \subseteq U$. Hence $U = X\setminus\set{x_0}$ is an open set which implies that $\set{x_0}$ is closed. Since $x_0 \in X$ was arbitrary, we conclude that $\ref{nbds} \implies \ref{1t1}$. 

%%%%%%%%%%%%%%%%%%%%%%%%% 16 %%%%%%%%%%%%%%%%%%%%%%%%%

\setcounter{subsection}{15} % 16
\subsection{}

\subsubsection{Problem}
Consider the five topologies on $\R$ given in Exercise 7 of \S 13.
\begin{enumerate}
    \item Determine the closure of the set $K := \{ 1/n \mid n \in \Z_+ \}$ under each of these topologies.
    \item Which of these topologies satisfy the Hausdorff axiom? The $T_1$ axiom?
\end{enumerate}
\subsubsection{Solution}
\todo

%%%%%%%%%%%%%%%%%%%%%%%%% 17 %%%%%%%%%%%%%%%%%%%%%%%%%

\setcounter{subsection}{16} % 17
\subsection{}

\subsubsection{Problem}
Consider the lower limit topology on $\R$ and the topology given by the basis $\mathcal{C}$ of Exercise 8 of \S 13. Determine the closures of the intervals $A := (0, \sqrt{2})$ and $B := (\sqrt{2}, 3)$ in these two topologies.
\subsubsection{Solution}
\todo

%%%%%%%%%%%%%%%%%%%%%%%%% 18 %%%%%%%%%%%%%%%%%%%%%%%%%

\setcounter{subsection}{17} % 18
\subsection{}

\subsubsection{Problem}
Determine the closures of the following subsets of the ordered square:
\begin{eqnarray*}
    A & := & \{ (1/n) \times 0 \mid n \in \Z_+ \},\\
    B & := & \{ (1-1/n) \times \tfrac12 \mid n \in \Z_+ \},\\
    C & := & \{ x \times 0 \mid 0 < x < 1 \},\\
    D & := & \{ x \times \tfrac12 \mid 0 < x < 1 \},\\
    E & := & \{ \tfrac12 \times y \mid 0 < y < 1 \}.
\end{eqnarray*}
\subsubsection{Solution}
\todo

%%%%%%%%%%%%%%%%%%%%%%%%% 19 %%%%%%%%%%%%%%%%%%%%%%%%%

\setcounter{subsection}{18} % 19
\subsection{}
\subsubsection{Problem}
If $A \subseteq X$, we define the \emph{boundary} of $A$ by the equation
\[\Bd A = \bar A \cap \overline{X \setminus A}\]
\begin{enumerate}
\item Show that $\Int A$ and $\Bd A$ are disjoint, and $\bar A = \Int A \cup \Bd A$.
\item Show that $\Bd A = \emptyset \iff A$ is both open and closed.
\item Show that $U$ is open $\iff \Bd U = \bar U \setminus U$.
\item If $U$ is open, is it true that $U = \Int(\bar U)$? Justify your answer.
\end{enumerate}

\subsubsection{Solution}
\begin{enumerate}
\item $X \setminus A$ is a subset of the closed set $X \setminus \Int A$, hence $\overline{X \setminus A} \subseteq X \setminus \Int A$. Thus $\Bd A \subseteq \overline{X \setminus A} \subseteq X \setminus \Int A$. $\Bd A$ and $\Int A$ are disjoint.

Let $x \in \bar A$. If every neighbourhood of $x$ intersects $X \setminus A$, then $x \in \overline{X \setminus A}$, and hence $x \in \Bd A$. Otherwise, there is some neighbourhood of $x$ which does not intersects $X \setminus A$, that is, it is a subset of $A$, and so $x \in \Int A$. Thus $\bar A = \Int A \cup \Bd A$.

\item If $\Bd A = \emptyset$, then $\bar A$ and $\overline{X \setminus A}$ form a partition of $X$. Thus $A = \bar A$ and $X \setminus A = \overline{X \setminus A}$. Both $A$ and $X \setminus A$ are closed.

Conversely, if $A$ is both open and closed, then $X \setminus A$ is closed as well, and so $\Bd A = A \cap (X \setminus A) = \emptyset$.

\item For all $A \subseteq X$, we know that $\Int A$ and $\Bd A$ form a partition of $\bar A$, and hence $\Bd A = \bar A \setminus \Int A$. Then, noting that $A$ and $\Int A$ are always subsets of $\bar A$,
\begin{align*}
& U \textrm{ is open} \\
\iff& \Int U = U \\
\iff& \bar U \setminus \Int U = \bar U \setminus U \\
\iff& \Bd U = \bar U \setminus U
\end{align*}

\item No. Let $X = \R$ with the standard topology, and let $U = (0,1) \cup (1,2)$.
\end{enumerate}

%%%%%%%%%%%%%%%%%%%%%%%%% 20 %%%%%%%%%%%%%%%%%%%%%%%%%

\setcounter{subsection}{19} % 20
\subsection{}

\subsubsection{Problem}
Find the boundary and the interior of each of the following subsets of $\R^2$:
\begin{enumerate}
    \item $A := \{ x \times y \mid y = 0 \}$
    \item $B := \{ x \times y \mid x > 0, y \neq 0 \}$
    \item $C := A \cup B$
    \item $D := \{ x \times y \mid x \in \Q \}$
    \item $E := \{ x \times y \mid 0 < x^2-y^2 \leq 1 \}$
    \item $F := \{ x \times y \mid x \neq 0, y \leq 1/x \}$
\end{enumerate}
\subsubsection{Solution}
\todo

%%%%%%%%%%%%%%%%%%%%%%%%% 21 %%%%%%%%%%%%%%%%%%%%%%%%%

\setcounter{subsection}{20} % 21
\subsection{}

\subsubsection{Problem}
(Kuratowski) Consider the collection of all subsets $A$ of the topological space $X$. The operations of closure $A \to \overline{A}$ and complementation $A \to X-A$ are functions from this collections to itself.
\begin{enumerate}
    \item Show that starting with a given set $A$, one can form no more than 14 distinct sets by applying these two operations successively.
    \item Find a subset $A$ of $\R$ (in its usual topology) for which the maximum of 14 is obtained.
\end{enumerate}
\subsubsection{Solution}
\todo


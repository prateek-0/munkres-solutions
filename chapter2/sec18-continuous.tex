\setcounter{section}{17} % 18
\section{Continuous functions}
\label{sec-topspace-continuous}

%%%%%%%%%%%%%%%%%%%%%%%%% 1 %%%%%%%%%%%%%%%%%%%%%%%%%

\subsection{}

\subsubsection{Problem}
Prove that for functions $f : \R \to \R$, the $\epsilon$-$\delta$ definition of continuity implies the open set definition.

\subsubsection{Solution}
Let us assume that $f$ is continuous by the $\epsilon$-$\delta$ definition of continuity. Let $V$ be an open set in $\R$. We need to show that $f^{-1}[V]$ is also open in $\R$. Let $x_0 \in f^{-1}[V] \implies f(x_0) \in V$. Since $V$ is open, there exists an interval $(a, b)$ such that $f(x_0) \in (a, b) \subseteq V$. Choose $\epsilon := \min\{\tfrac{x_0-a}{2}, \tfrac{b-x_0}{2}\}$. By the $\epsilon$-$\delta$ definition, there exists $\delta > 0$ such that for all $x$ such that $|x-x_0| < \delta$ we have that $|f(x)-f(x_0)| < \epsilon$. This implies that $f[(x-\delta, x+\delta)] \subseteq (f(x)-\epsilon, f(x)+\epsilon) \subseteq V$. Hence for each $x$ in $f^{-1}[V]$, there exists an open set $U_x := (x-\delta, x+\delta)$ such that $x \in (x-\delta, x+\delta) \subseteq f^{-1}[V]$. By Chapter 2, \S 13, Problem 1 we conclude that $f^{-1}[V]$ is open in $\R$. Hence we get that $f$ is continuous by the open set definition of continuity.

%%%%%%%%%%%%%%%%%%%%%%%%% 2 %%%%%%%%%%%%%%%%%%%%%%%%%

\subsection{}

\subsubsection{Problem}
Suppose that $f : X \to Y$ is continuous. If $x$ is a limit point of the subset $A$ of $X$, is it necessarily true that $f(x)$ is a limit point of $f[A]$?

\subsubsection{Solution}
The above statement is not true. Consider $f : \R \to \R$ given by $f(x) \equiv 0$. First we observe that $f$ is continuous; clearly for any set $U \subseteq \R$ such that $0 \not\in U$, $f^{-1}[U] = \emptyset$ which is open, and for any set $V \subseteq \R$ such that $0 \in V, f^{-1}[V] = \R$ which is also open. Finally we see that $\pi \in \R$ is a limit point of $\R$, but $f(\pi) = 0$ is not a limit point of $f[\R] = \{0\}$, since $\{0\}$ has no limit points.

%%%%%%%%%%%%%%%%%%%%%%%%% 3 %%%%%%%%%%%%%%%%%%%%%%%%%

\subsection{}

\subsubsection{Problem}
Let $X$ and $X'$ denote a single set in the two topologies $\T$ and $\T'$ respectively. Let $i : X' \to X$ be the identity function. 
\begin{enumerate}
    \item Show that $i$ is continuous $\iff$ $\T'$ is finer than $\T$.
    \item Show that $i$ is a homeomorphism $\iff \T' = \T$.
\end{enumerate}

\subsubsection{Solution}
\begin{enumerate}
    \item We have that $i$ is continuous $\iff$ $A = i^{-1}[A] \in \T'$ for every open set $A \in \T \iff \T \subseteq \T' \iff \T'$ is finer than $\T$.
    \item Using the result in part (a) and the fact that $i$ is already a bijection, we have that $i$ is a homeomorphism $\iff$ $i : X' \to X, i^{-1} : X \to X'$ are both continuous $\iff$ $\T'$ is finer than $\T$ and $\T$ is finer than $\T' \iff \T' = \T$.
\end{enumerate}

%%%%%%%%%%%%%%%%%%%%%%%%% 4 %%%%%%%%%%%%%%%%%%%%%%%%%

\subsection{}

\subsubsection{Problem}
Given $x_0 \in X$ and $y_0 \in Y$, show that the maps $f : X \to X \times Y$ and $g : Y \to X \times Y$ defined by
\[ f(x) := x \times y_0 ~~~~~\mbox{ and }~~~~~ g(y) := x_0 \times y \]
are imbeddings.

\subsubsection{Solution}
\todo

%%%%%%%%%%%%%%%%%%%%%%%%% 5 %%%%%%%%%%%%%%%%%%%%%%%%%

\subsection{}

\subsubsection{Problem}
Show that the subspace $(a, b)$ of $\R$ is homeomorphic with $(0, 1)$ and the subspace $[a, b]$ of $\R$ is homeomorphic with $[0, 1]$.

\subsubsection{Solution}
\todo

%%%%%%%%%%%%%%%%%%%%%%%%% 6 %%%%%%%%%%%%%%%%%%%%%%%%%

\setcounter{subsection}{5} % 6
\subsection{}

\subsubsection{Problem}
Find a function $f : \R \to \R$ that is continuous at precisely one point.

\subsubsection{Solution}
Consider the function $f : \R \to \R$ defined by
\[
    f(x) := \left\{ \begin{array}{ll}
        x & \mbox{ if } x \in \Q, \\
        0 & \mbox{ if } x \in \Q' := \R\setminus\Q.
    \end{array} \right.
\]
We claim that this function is only continuous at $x = 0$ and nowhere else. Since we know from Problem 1 that the $\epsilon$-$\delta$ definition of continuity is equivalent to the open set definition, we will use the former to show that $f$ is continuous.

To show that $f$ is continuous at 0, let $\epsilon > 0$ be given. Pick $\delta := \tfrac{\epsilon}{2}$. Then we see that for all $x \in B_\delta(0) \cap \Q$ we have $|f(x)| = |x| < \delta < \epsilon$. Also for all $x \in B_\delta(0) \cap \Q'$ we have $|f(x)| = 0 < \epsilon$. Hence we see that $f$ is continuous at 0.

We show that $f$ is not continuous at any other point by contradiction. If possible let $f$ be continuous at $0 \neq x_0 \in \R$. Choose $\epsilon := \tfrac{x_0}{3} > 0$.

If $x_0 \in \Q$, then we have $f(x_0) = x_0$. For this $\epsilon$, there exists $\delta > 0$ such that for all $x \in (x_0-\delta, x_0+\delta)$, $f(x) \in (x_0-\epsilon, x_0+\epsilon) = \left( \tfrac{2x_0}{3}, \tfrac{4x_0}{3} \right)$. But since we know that $\Q'$ is dense in $\R$, there exists $q' \in \Q'$ such that $q' \in (x_0-\delta, x_0+\delta) \implies 0 = f(q') \in \left( \tfrac{2x_0}{3}, \tfrac{4x_0}{3} \right)$, a contradiction.

Similarly if $x_0 \in \Q'$, we have $f(x_0) := 0$. For this $\epsilon$, there exists $\delta > 0$ such that for all $x \in (x_0-\delta, x_0+\delta)$, $f(x) \in (-\epsilon, \epsilon) = \left( -\tfrac{x_0}{3}, \tfrac{x_0}{3} \right)$. But since we know that $\Q$ is dense in $\R$, there exists $q \in Q$ such that $q \in (x_0-\delta, x_0+\delta) \cap (x_0-\epsilon, x_0+\epsilon)$. Then we have that $q = f(q) \in \left( -\tfrac{x_0}{3}, \tfrac{x_0}{3} \right)$, which contradicts the fact that $q \in (x_0-\epsilon, x_0+\epsilon) = \left( \tfrac{2x_0}{3}, \tfrac{4x_0}{3} \right)$.

Hence $f$ is not continuous at any point in $\R$ except $0$.

%%%%%%%%%%%%%%%%%%%%%%%%% 7 %%%%%%%%%%%%%%%%%%%%%%%%%

\subsection{}

\subsubsection{Problem}
\begin{enumerate}
    \item Suppose that $f : \R \to \R$ is ``continuous from the right'', that is,
    \[ \lim_{x \to a^+}f(x) = f(a) \]
    for each $a \in \R$. Show that $f$ is continuous when considered as a function from $\R_\ell$ to $\R$.
    \item Can you conjecture what functions $f : \R \to \R$ are continuous when considered as maps from $\R$ to $\R_\ell$? As maps from $\R_\ell$ to $\R_\ell$? We shall return to this question in Chapter 3.
\end{enumerate}

\subsubsection{Solution}
\todo

%%%%%%%%%%%%%%%%%%%%%%%%% 8 %%%%%%%%%%%%%%%%%%%%%%%%%

\subsection{}

\subsubsection{Problem}
Let $Y$ be an ordered set in the order topology. Let $f, g : X \to Y$ be continuous.
\begin{enumerate}
    \item Show that the set $\{ x \mid f(x) \leq g(x) \}$ is closed in $X$.
    \item Let $h : X \to Y$ be the function $h(x) := \min\{ f(x), g(x) \}$. Show that $h$ is continuous.
\end{enumerate}

\subsubsection{Solution}
\todo

%%%%%%%%%%%%%%%%%%%%%%%%% 9 %%%%%%%%%%%%%%%%%%%%%%%%%

\subsection{}

\subsubsection{Problem}
Let $\{ A_\alpha \}$ be a collection of subsets of $X$; let $X = \bigcup_\alpha A_\alpha$. Let $f : X \to Y$; suppose that $f \mid A_\alpha$ is continuous for each $\alpha$.
\begin{enumerate}
    \item Show that if the collection $\{A_\alpha\}$ is finite and each $A_\alpha$ is closed, then $f$ is continuous.
    \item Find an example where the collection $\{A_\alpha\}$ is countable and each $A_\alpha$ is closed, but $f$ is not continuous.
    \item An indexed family of sets $\{A_\alpha\}$ is said to be \textbf{locally finite} if each point $x$ of $X$ has a neighbourhood that intersects $A_\alpha$ for only finitely many values of $\alpha$. Show that if the family $\{A_\alpha\}$ is locally finite and each $A_\alpha$ is closed, then $f$ is continuous.
\end{enumerate}

\subsubsection{Solution}
\todo

%%%%%%%%%%%%%%%%%%%%%%%%% 10 %%%%%%%%%%%%%%%%%%%%%%%%%

\subsection{}

\subsubsection{Problem}
Let $f : A \to B$ and $g : C \to D$ be continuous functions. Let us define a map $f \times g : A \times C \to B \times D$ by the equation
\[ (f \times g)(a \times c) := f(a) \times g(c),~ \forall a \in A, c \in C \]
Show that $f \times g$ is continuous.

\subsubsection{Solution}
\todo

%%%%%%%%%%%%%%%%%%%%%%%%% 11 %%%%%%%%%%%%%%%%%%%%%%%%%

\subsection{}

\subsubsection{Problem}
Let $F : X \times Y \to Z$. We say that $F$ is \textbf{continuous in each variable separately} if for each $y_0$ in $Y$, the map $h : X \to Z$ defined by $h(x) := F(x \times y_0)$ is continuous, and for each $x_0$ in $X$, the map $k : Y \to Z$ defined by $k(y) := F(x_0 \times y)$ is continuous. Show that if $F$ is continuous, then F is continuous in each variable separately.

\subsubsection{Solution}
\todo

%%%%%%%%%%%%%%%%%%%%%%%%% 12 %%%%%%%%%%%%%%%%%%%%%%%%%

\subsection{}

\subsubsection{Problem}
Let $F : \R \times \R \to \R$ be defined by the equation
\[
F(x \times y) := 
\left\{ \begin{array}{ll}
    xy/(x^2+y^2) & \mbox{ if } x \times y \neq 0 \times 0 ; \\
    0 & \mbox{ if } x \times y = 0 \times 0.
\end{array} \right.
\]
\begin{enumerate}
    \item Show that $F$ is continuous in each variable separately.
    \item Compute the function $g : \R \to \R$ defined by $g(x) := F(x \times x)$ for all $x \in \R$.
    \item Show that $F$ is not continuous.
\end{enumerate}

\subsubsection{Solution}
\todo

%%%%%%%%%%%%%%%%%%%%%%%%% 13 %%%%%%%%%%%%%%%%%%%%%%%%%

\subsection{}

\subsubsection{Problem}
Let $A \subseteq X$; let $f : A \to Y$ be continuous; let $Y$ be Hausdorff. Show that if $f$ may be extended to a continuous function $g : \bar A \to Y$, then $g$ is uniquely determined by $f$.

\subsubsection{Solution}
Let $g_1, g_2 : \bar A \to Y$, which agree on $A$. We have to show that $g_1 = g_2$. Let
\[h : \bar A \to Y \times Y,\ h(x) = g_1(x) \times g_2(x) \]
Consider the diagonal $\Delta = \{ y \times y : y \in Y\}$. Since $Y$ is Hausdorff, $\Delta$ is closed. As $h$ is continuous, $h^{-1}(\Delta)$ is closed. It includes $A$, and hence includes $\bar A$.

\section*{Supplementary Exercises: Topological Groups} 
\addcontentsline{toc}{section}{\numberline {\S *}Supplementary Exercises: Topological Groups}
\label{sec-topspace-groups}

%%%%%%%%%%%%%%%%%%%%%%%%% 3 %%%%%%%%%%%%%%%%%%%%%%%%%

\setcounter{subsection}{2} % 3
\subsection{}

\subsubsection{Problem}
Let $H$ be a subspace of $G$. Show that if $H$ is also a subgroup of $G$, then both $H$ and $\bar H$ are topological groups.

\subsubsection{Solution}
Since $H$ is a subspace of a $T_1$ space, $H$ is $T_1$ too. The multiplication and inverse functions on $H$ are obtained by restricting the domain and codomain of the ones on $G$, and so are continuous. So $H$ is a topological group. In particular, every subgroup of $G$ with the subspace topology is a topological group.

Let $m : G \times G \to G$ and $i : G \to G$ be the multiplication and inverse functions respectively. Using basic group properties, $i(H) = H$ and $m(H \times H) = H$. Using the continuity of $m$ and $i$,
\begin{itemize}
\item $i(\bar H) \subseteq \overline{i(H)} = \bar H$.
\item $m(\bar H \times \bar H) = m(\overline{H \times H}) \subseteq \overline{m(H \times H)} = \bar H$.
\end{itemize}
$\bar H$ contains the identity and is closed under inverses and multiplication, so $\bar H$ is a subgroup of $G$. Thus $\bar H$ is a topological group.

%%%%%%%%%%%%%%%%%%%%%%%%% 5 %%%%%%%%%%%%%%%%%%%%%%%%%

\setcounter{subsection}{4} % 5
\subsection{}

\subsubsection{Problem}
Let $H$ be a subgroup of $G$. If $x \in G$, define $xH = \{xh : h \in H\}$; this set is called a \emph{left coset} of $H$ in $G$. Let $G/H$ be the collection of left cosets of $H$ in $G$; it is a partition of $G$. Give $G/H$ the quotient topology.
\begin{enumerate}
\item Show that if $\alpha \in G$, the map $f_\alpha$ of the preceding exercise induces a homeomorphism of $G/H$ carrying $xH$ to $(\alpha x)H$. Conclude that $G/H$ is a homogeneous space.
\item Show that if $H$ is a closed set in the topology of $G$, then one-point sets are closed in $G/H$.
\item Show that the quotient map $p : G \to G/H$ is open.
\item Show that if $H$ is closed in the topology of $G$ and is a normal subgroup of $G$, then $G/H$ is a topological group.
\end{enumerate}

\subsubsection{Solution}
\begin{enumerate}
\item $f_\alpha : G \to G$ given by $f_\alpha(x) = \alpha x$ is an homeomorphism from $G$ to $G$.

\begin{center}
\begin{tikzcd}
G \arrow[r, "f_\alpha"] \arrow[d, "p"] & G \arrow[d, "p"] \\
G/H \arrow[r, dashed] & G/H
\end{tikzcd}
\end{center}
We use Theorem 22.2. $p : G \to G/H$ is a quotient map, and $p \circ f_\alpha$ is a map which is constant on each $p^{-1}(\{y\})$. This is because if $g_1,g_2 \in p^{-1}(\{y\})$, then $g_1 = xh_1$ and $g_2 = xh_2$ for some $x \in G$. Then 
\[p(f_\alpha(g_1)) = p(\alpha x h_1) = \alpha x H = p(\alpha x h_2) = p(f_\alpha(g_2))\]
Note that $p \circ f_\alpha$ is a quotient map, being the composite of a quotient map and a homeomorphism. From Theorem 22.2, a map from $G/H$ to $G/H$ which maps $xH$ to $\alpha xH$ is induced, which is a quotient map too. It's easy to see that this is a bijection, with the inverse given by a similar map which maps $xH$ to $\alpha^{-1} xH$. Being a bijective quotient map, it is a homeomorphism.

\item If $H$ is closed, so is every $xH$, since left-multiplication by $x$ is a homeomorphism from $G$ to $G$. Thus every one-point set in $G/H$ is closed.

\item Let $U \subseteq G$ be open. We need to show that $p(U)$ is open, which is equivalent to $p^{-1}(p(U))$ being open.
\begin{align*}
p^{-1}(p(U)) &= p^{-1}(\{p(u) : u \in U \}) \\
&= p^{-1}(\{uH : u \in U \}) \\
&= \bigcup_{u \in U} uH \\
&= \{uh : u \in U, h \in H \} \\
&= \bigcup_{h \in H} Uh
\end{align*}
Since right-multiplication by $h$ is a homeomorphism, every $Uh$ is open, and hence their union is open.

\item Since $H$ is closed, $G/H$ is $T_1$. Consider the map $p \times p : G \times G \to G/H \times G/H$. Being the product of two open maps, it is an open map, and hence a quotient map. Consider $f : G \times G \to G$ given by $f(x \times y) = xy^{-1}$. We know that $f$ is continuous. Consider $f_H : G/H \times G/H \to G/H$ also given by $f_H(x \times y) = xy^{-1}$, with the operations in the group $G/H$. We need to show that $f_H$ is continuous. From basic group theory, the following diagram commutes:
\begin{center}
\begin{tikzcd}
G\times G \arrow[r, "f"] \arrow[d, "p \times p"] & G \arrow[d, "p"] \\
G/H \times G/H \arrow[r, "f_H"] & G/H
\end{tikzcd}
\end{center}
We apply Theorem 22.2, using that $p \times p$ is a quotient map and that $p \circ f$ is continuous. Thus $f_H$ is continuous. $G/H$ is a topological group.

\end{enumerate}

%%%%%%%%%%%%%%%%%%%%%%%%% 7 %%%%%%%%%%%%%%%%%%%%%%%%%

\setcounter{subsection}{6} % 7
\subsection{}

\subsubsection{Problem}
If $A$ and $B$ are subsets of $G$, let $AB$ denote the set of points $ab$ for $a \in A$ and $b \in B$. Let $A^{-1}$ denote the set of all points $a^{-1}$, for $a \in A$.
\begin{enumerate}
\item A neighbourhood $V$ of the identity element is said to be \emph{symmetric} if $V = V^{-1}$. If $U$ is a neighbourhood of $e$, show that there is a symmetric neighbourhood $V$ of $e$ such that $VV \subseteq U$.
\item Show that $G$ is Hausdorff. In fact, show that if $x \neq y$, there is a neighbourhood $V$ of $e$ such that $Vx$ and $Vy$ are disjoint.
\item \label{topgroup-p7-c} Show that $G$ satisfies the following separation axiom, called the \emph{regularity axiom}: Given a closed set $A$ and a point $x$ not in $A$, there exist disjoint open sets including $A$ and containing $x$ respectively. [Hint: There is a neighbourhood $V$ of $e$ such that $Vx$ and $VA$ are disjoint.]
\item  Let $H$ be a subgroup of $G$ that is closed in the topology of $G$; let $p : G \to G/H$ be the quotient map. Show that $G/H$ satisfies the regularity axiom. [Hint: Examine the proof of \ref{topgroup-p7-c} when $A$ is saturated.]
\end{enumerate}

\subsubsection{Solution}
\begin{enumerate}
\item Consider $m : G \times G \to G$, the multiplication function. $m^{-1}(U)$ is open in $G$ and contains $e \times e$, so there exist $V_1$, $V_2$ open in $G$ such that $e \times e \in V_1 \times V_2 \subseteq m^{-1}(U)$. Let $V = V_1 \cap V_2 \cap V_1^{-1} \cap V_2^{-1}$. Since the inverse function is a homeomorphism from $G$ to $G$, $V_1^{-1}$ and $V_2^{-1}$ are open. $V$ is symmetric, open, and contains $e$. $VV \subseteq V_1V_2 \subseteq U$.
\item Suppose $x, y \in G$ are distinct. Then $G \setminus \{xy^{-1}\}$ is a neighbourhood of the identity $e$, so there exists a symmetric neighbourhood $V$ of $e$ such that $VV \subseteq G \setminus \{xy^{-1}\}$. Then $Vx$ and $Vy$ are open, because for any $v_1, v_2 \in V$, $v_1x \neq v_2y$, as $xy^{-1} \neq v_2v_1^{-1}$. Since right-multiplication by a given element is a homeomorphism of $G$ to itself, $Vx$ and $Vy$ are open. They contain $x$ and $y$ respectively, because $e \in V$.
\item The proof is similar to the above. $Ax^{-1}$ is closed, so there exists a symmetric neighbourhood $V$ of $e$ such that $VV \subseteq G \setminus Ax^{-1}$. Then $VA$ and $Vx$ are disjoint, because for all $v_1, v_2 \in V$ and $a \in A$, $v_1a \neq v_2x$, as $ax^{-1} \neq v_2v_1^{-1}$. $Vx$ is open, and $VA = \bigcup_{a \in A} Va$ is open too. We have $x \in Vx$ and $A \subseteq VA$, because $e \in V$.

\item Consider $y \in G/H$ and $B \subseteq G/H$ such that $y \notin B$ and $B$ is closed. Then $p^{-1}(\{y\})$ is a coset $xH$ for some $x \in G$, and $p^{-1}(B)$ is a union of cosets, say $AH$ for some $A \subseteq G$. Since $p$ is a quotient map, $AH$ is closed. $xH$ and $AH$ are disjoint. In particular, $e \notin AHx^{-1}$, and $AHx^{-1}$ is closed.

Thus there is a symmetric neighbourhood $V$ of $e$ such that $VV \subseteq G \setminus AHx^{-1}$. Then $VAH$ and $VxH$ are disjoint, because for all $v_1, v_2 \in V$, $a \in A$, and $h_1, h_2 \in H$, $v_1ah_1 \neq v_2xh_2$, as $ah_1h_2^{-1}x^{-1} \neq v_2v_1^{-1}$. $VAH$ is open, because it is $\bigcup_{q \in AH} Vq$, and similarly $VxH$ is open too. $AH \subseteq VAH$ and $xH \subseteq VxH$, because $e \in V$. Both $VAH$ and $VxH$ are unions of cosets, hence are saturated with respect to $p$. So their images $p(VAH)$ and $p(VxH)$ are open in $G/H$, disjoint, and include $A$ and $\{y\}$ respectively.
\end{enumerate}

